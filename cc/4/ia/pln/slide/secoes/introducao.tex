\renewcommand{\titulo}{Introdução}
\section{\titulo}

\begin{frame}{Sumário}
\tableofcontents[currentsection]
\end{frame}

\renewcommand{\titulo}{Definição}
\subsection{\titulo}
\begin{frame}{\titulo}
Processamento de Linguagem Natural (PLN) é uma sub-área da inteligência artificial e da linguística que estuda os problemas de geração e compreensão automática de linguas humanas naturais.
\end{frame}


\renewcommand{\titulo}{Características da Linguagem}
\subsection{\titulo}

\begin{frame}{\titulo}
As dificuldades que existem ao tentar analisar a linguagem é apenas o lado menos interessante das propriedades que tornam a linguagem tão poderosa.
\end{frame}


\begin{frame}{Significado vago}
\begin{itemize}
\item \textbf{Problemas:} As frases de uma língua são descrições incompletas das informações que pretendem transmitir
\item \textbf{Ponto positivo:} A linguagem permite aos que falam ser tão vagos ou prescisos quanto quiserem, também permite que deixem de mencionar coisas que acham que os ouvintes já sabem.
\end{itemize}
\end{frame}

\begin{frame}{Significado vago - exemplo}
\begin{table}[!htp]
	\begin{tabular}{| p{5.25cm} | p{5.25cm} |}

		\hline \textbf{Há alguns cachorros lá fora.} & \textbf{Liguei para Linda e convidei-a para ir ao cinema.} \\ 
													 & \textbf{Ela disse que gostaria muito.} \\
		\hline Há alguns cachorros no jardim. & Linda estava em casa quando eu telefonei. \\
		\hline Há três cachorros no jardim. & Ela atendeu o telefone. \\
		\hline Fido, Tobi e Bibba estão no jardim. & Eu fiz o convite. \\
		\hline

	\end{tabular}
\end{table}
\end{frame}

\begin{frame}{Sensibilidade ao contexto}
\begin{itemize}
\item \textbf{Problemas:} A mesma expressão significa coisas diferentes em contextos diferentes.
\item \textbf{Ponto positivo:} A linguagem nos permite dizer coisas sobre um mundo infinito usando um número finito de símbolos.
\end{itemize}
\end{frame}

\begin{frame}{Sensibilidade ao contexto - exemplo}
\begin{table}[!htp]
	\begin{tabular}{| p{3cm} | p{7.5cm} |}

		\hline Como está a água? & Em um laboratório químico, ela prescisa ser pura \\
		\hline Como está a água? & Se você está com sede, ela prescisa ser potável \\
		\hline Como está a água? & Ao lidar com goteiras no telhado, ela pode estar imunda \\
		\hline

	\end{tabular}
\end{table}
\end{frame}


\begin{frame}{Dinamicidade da linguagem}
\begin{itemize}
\item \textbf{Problemas:} Nenhum programa de linguagem natural pode ser completo porque novas palavras, expressões e significados podem ser gerados com bastante liberdade.
\item \textbf{Ponto positivo:} A linguagem pode evoluir juntamente com a evolução das experiências que queremos comunicar.
\end{itemize}
\end{frame}

\begin{frame}{Dinamicidade da linguagem - exemplo}
\begin{block}{}
\begin{itemize}
\item Eu xeroco uma cópia para você.
\item Vou googlar a respeito de processamento de linguagem natural.
\end{itemize}
\end{block}
\end{frame}


\begin{frame}{Ambiguidade}
\begin{itemize}
\item \textbf{Problemas:} Há inumeras maneiras de dizer a mesma coisa.
\item \textbf{Ponto positivo:} Quando se sabe muito, os fatos simplesmente estão implícitos uns nos outros. A linguagem é para ser usada por agentes que sabem muito.
\end{itemize}
\end{frame}

\begin{frame}{Ambiguidade - exemplo}
\begin{block}{}
\begin{itemize}
\item Maria nasceu no dia 11 de outubro.
\item O aniversário de Maria é no dia 11 de outubro.
\end{itemize}
\end{block}
\end{frame}

\renewcommand{\titulo}{Áreas que auxiliam o PLN}
\subsection{\titulo}
\begin{frame}{\titulo}
\begin{itemize}
\item Ciência da Computação
\item Linguística
\item Ciências Cognitivas
\item Fonética
\item Inteligência Artificial
\item Reconhecimento de Padrões
\end{itemize}
\end{frame}

\renewcommand{\titulo}{Sub-áreas de PLN}
\subsection{\titulo}
\begin{frame}{\titulo}
\begin{itemize}
\item \textbf{Interpretação de linguagem natural:} baseia-se em mecanismos que tentam ``compreender'' frases em alguma linguagem natural, buscando traduzi-las para uma representação que possa ser compreendida e utilizada pelo computador (ex: lógica de predicados).
\item \textbf{Geração de linguagem natural:} o computador traduz uma representação interna de um conteúdo semântico pré definido para uma expressão em alguma língua.
\end{itemize}
\end{frame}

\renewcommand{\titulo}{Chomsky (1950)}
\subsection{\titulo}
\begin{frame}{\titulo}
\emph{``Existe por trás da língua, de um modo não palpável, um corpo de generalizações, princípios e regras abstratas em número finito, que determinam as frases da língua, a sua gramaticalidade, suas propriedades e características. Este corpo altamente organizado chama-se gramática. Cada ser humano possui então uma gramática interiorizada adquirida enquanto	criança num período relativamente curto e possivelmente	na base de alguns princípios inatos, próprios à espécie	humana, a faculdade da linguagem.''}
\end{frame}


\renewcommand{\titulo}{Hierarquia de Chomsky}
\subsection{\titulo}
\begin{frame}{\titulo}
\figura{./img/chomsky}{0.22}
\end{frame}


