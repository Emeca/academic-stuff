\renewcommand{\titulo}{Finalização}
\section{\titulo}

\begin{frame}{Sumário}
\tableofcontents[currentsection]
\end{frame}

\renewcommand{\titulo}{Conclusão}
\subsection{\titulo}
\begin{frame}{\titulo}
Processamento de Linguagem Natural está apenas começando, se trata de uma área bastante promissora para um futuro que precisa de inteligência humana simulada para executar tarefas que	apenas o ser humano até agora pode resolver satisfatoriamente.
\end{frame}

\renewcommand{\titulo}{Perguntas e Discussão}
\subsection{\titulo}
\begin{frame}{\titulo}
\begin{center}
%\vspace{1cm}
Eder Ruiz Maria \\
eder@gnoia.org \\
\vspace{0.5cm}
Paulo Leonardo Benatto \\
patito@gnoia.org
\end{center}
\end{frame}

\renewcommand{\titulo}{Referências}
\subsection{\titulo}
\begin{frame}{\titulo}
\begin{itemize}
\item BARROS, F. A.; ROBIN J. \textbf{Processamento de Linguagem Natural}
\item SCHNEIDER, M. O. \textbf{Processamento de Linguagem Natual (PLN)}
\item RICH, E.; KNIGHT, K. \textbf{Inteligência Artificial}. 2 ed.
\item RUSSEL, S.; NORVIG, P. \textbf{Inteligência Artificial}. 2 ed.
\item http://www.inf.pucrs.br/\~{}linatural
\end{itemize}
\end{frame}


