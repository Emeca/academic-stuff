\renewcommand{\titulo}{Métodos de Aquisição de Conhecimento}
\section{\titulo}
\begin{frame}
Para a aquisição do conhecimento foram utilizados os seguintes métodos:
\begin{block}{Métodos utilizados}
	\begin{itemize}
	\item Brainstorming
	\pause
	\item Entrevista desestruturada
	\pause
	\item Entrevista estrutura
	\end{itemize}
\end{block}
\end{frame}

\renewcommand{\titulo}{Brainstorming}
\subsection{\titulo}
\begin{frame}{\titulo}
Para o desenvolvimento do trabalho de sistema baseado em conhecimento, primeiramente foi realizado um brainstorming com os engenheiros do ITAI, para coletar idéias interessantes, assim escolhendo um tema.
\end{frame}


\renewcommand{\titulo}{Entrevista Desestruturada}
\subsection{\titulo}
\begin{frame}{\titulo}
A entrevista desestruturada foi utilizada na primeira reunião, onde o objetivo do grupo é explorar o domínio do problema.
\begin{block}{}
	\begin{itemize}
	\item Funcionamento de energia elétrica em uma residência
	\item Principais componentes
	\item Principais falhas
	\end{itemize}
\end{block}
\end{frame}

\renewcommand{\titulo}{Entrevista Estruturada}
\subsection{\titulo}
\begin{frame}{\titulo}
Para a segunda reunião foi utilizada uma entrevista estruturada, onde o foco do grupo era refinar a base de conhecimento utilizando perguntas específicas, de acordo com o interesse do grupo, não saindo do foco.
\begin{block}{}
	\begin{itemize}
	\item Processo detalhado do fornecimento de energia
	\item Função de cada componente 
	\end{itemize}
\end{block}
\end{frame}


