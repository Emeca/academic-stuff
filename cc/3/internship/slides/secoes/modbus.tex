\renewcommand{\titulo}{Modbus}
\section{\titulo}

\subsection{Histórico e Modelos}
\begin{frame}{Histórico}
\begin{itemize}
\item Sinal analógico de 4-20 mA (1960);
\item Fieldbus (1975-1980);
\item Modbus (1979);
\end{itemize}
\end{frame}

\renewcommand{\titulo}{Modelos}
\begin{frame}{\titulo}
\begin{itemize}
\item Modbus TCP/IP;
\item Modbus Plus;
\item Modbus Padrão;
\end{itemize}
\end{frame}

\renewcommand{\titulo}{Paradigma Mestre-Escravo}
\subsection{\titulo}
\begin{frame}{\titulo}
\begin{itemize}
\item utilizado em ambiente industrial;
\item todas as estações dependem de um mestre;
\item o mestre decide quando cada um dos escravos tem o direito de transmitir;
\item escravo = servidor;
\item mestre = cliente;
\end{itemize}
\end{frame}

\subsection{Frame e Transação}
\begin{frame}{\textit{Frame}}
\figura{\textit{Frame} Modbus}{frame}{img/frame}{0.3}
\end{frame}
\begin{frame}{Transação}
\figura{Transação Modbus}{transacao}{img/transacao}{0.3}
\end{frame}

\renewcommand{\titulo}{Hierarquia de Camadas}
\subsection{\titulo}
\begin{frame}{\titulo}
\figura{Hierarquia de Camadas}{camadas}{img/camadas}{0.3}
\end{frame}

\renewcommand{\titulo}{Modelo de Rede}
\subsection{\titulo}
\begin{frame}{\titulo}
\figura{Modelo de Rede}{rede}{img/rede}{0.25}
\end{frame}
